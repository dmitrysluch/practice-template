% Это основная команда, с которой начинается любой \LaTeX-файл. Она отвечает за тип документа, с которым связаны основные правил оформления текста.
\documentclass{article}

% Здесь идет преамбула документа, тут пишутся команды, которые настраивают LaTeX окружение, подключаете внешние пакеты, определяете свои команды и окружения. В данном случае я это делаю в отдельных файлах, а тут подключаю эти файлы.

% Здесь я подключаю разные стилевые пакеты. Например возможности набирать особые символы или возможность компилировать русский текст. Подробное описание внутри.
\usepackage{packages}

% Здесь я определяю разные окружения, например, теоремы, определения, замечания и так далее. У этих окружений разные стили оформления, кроме того, эти окружения могут быть нумерованными или нет. Все подробно объяснено внутри.
\usepackage{environments}

% Здесь я определяю разные команды, которых нет в LaTeX, но мне нужны, например, команда \tr для обозначения следа матрицы. Или я переопределяю LaTeX команды, которые работают не так, как мне хотелось бы. Типичный пример мнимая и вещественная часть комплексного числа \Im, \Re. В оригинале они выглядят не так, как мы привыкли. Кроме того, \Im еще используется и для обозначения образа линейного отображения. Подробнее описано внутри.
\usepackage{commands}

% Пакет для титульника исследовательского проекта
\usepackage{titlepage}



\setGroup{198}
%сюда можно воткнуть картинку подписи
\setStudentSgn{\includegraphics[height=32pt]{sign2.png}}
\setStudent{Д. Б. Случ}
\setStudentDate{03.06.2022}
\setAdvisor{Подольский Владимир Владимирович}
\setAdvisorTitle{Департамент больших данных и информационного поиска, доцент, д. ф.-м. н.}

\setYear{2022}




% С этого момента начинается текст документа
\begin{document}

% Эта команда создает титульную страницу
\makeTitlePage

% Здесь будет автоматически генерироваться содержание документа


% Данное окружение оформляет аннотацию: краткое описание текста выделенным абзацем после заголовка
\begin{abstract}
    Практика была пройдена в международной лаборатории теоретической информатики факультета компьютерных наук НИУ ВШЭ.    
\end{abstract}
\tableofcontents
% \section{Введение}

% \subsection{Результаты}
\section{Цель практики}
Получение опыта работы в лаборатории МЛ ТИ,
исследование коммуникационной сложности функций,
а также сложности в модели разрешающих деревьев
\section{Задачи практики}
\begin{enumerate}
    \item Закрепление и углубление теоретических знаний в
    области коммуникационной сложности
    \item Изучение статей в области односторонней и
    двусторонней коммуникационной сложности
    \item Исследовать одностороннюю и двустороннюю
    коммуникационную сложность композиций функций
    \item Изучение метрик сложности функций: чувствительности, блочной чувствительности, сертификатной сложности, сложности в модели разрешающих деревьев
    \item Исследование связи между метриками сложности определенных классов функций
\end{enumerate}
\section{Изученные материалы}
\begin{enumerate}
        \item Bruno Loff and Sagnik Mukhopadhyay. Lifting Theorems for Equality.\cite{loff}
\end{enumerate}

% \begin{definition}
% Назовем 0-сложностью дерева решений, максимальное по выбору листа количество запросов на которые ответили 0 по пути из корня в лист дерева. 0-сложностью функции в модели решающих деревьев $D^{dt}_{0}(f)$ назовем минимальную по выбору дерева 0-сложность дерева. Аналогично определяется 1 сложность функции $D^{dt}_{1}(f)$.
% \end{definition}
% \begin{lemma}
%     Рассмотрим два дерева решений для одной функции. Рассмотрим путь $a$ в первом дереве - лист в который он ведет помечен $0$. Тогда для любого пути $b$ во втором дереве в лист помеченный $1$ существует индекс $i$, такой, что $a$ и $b$ делают запрос к индексу $i$, при этом у пути $a$ и $b$ разные значения на этом запросе.
%     \begin{proof}Допустим противное. Тогда существует строка, если мы пойдем по которой в первом дереве мы придем в 0, а во втором в 1. Действительно положим ее равной значениям $a$ на индексах, которые запрашиваются на пути $a$ и значениям $b$ на индексах, которые запрашивает $b$. 
%     \end{proof}
% \end{lemma}
% \begin{theorem}
%     $D^{dt}(f) = O(D^{dt}_0(f)D^{dt}_1(f))$
%     \begin{proof}
%         Рассмотрим оптимальные по $D^{dt}_0(f)$ и $D^{dt}_1(f)$ решающие деревья $A$ и $B$. Максимальная глубина по единичкам листа, который отвечает $0$ это $d_0 \leq D^{dt}_1(f)$, а максимальная по единичкам глубина листа, который отвечает $1$ это $d_1 \leq D^{dt}_1(f)$.\\
%         Запустим следующий алгоритм. В дереве $A$ смотрим на путь, на котором одни нули. Без потери общности ответ на этом пути 0. Этот путь пересекается со всеми путями в $B$ на которых ответ 1, при этом значение этих путей на пересечении 1(т.к. путь из одних нулей). Сделаем все запросы на этом пути. Это займет $D^{dt}_0$ операций. Если строка им удовлетворяет - мы знаем значение функции - победа. Если не удовлетворяет, то т.к. пути пересекаются мы узнали по минимум одному биту на каждом пути с листом помеченным 1 в $B$. Теперь мы знаем некоторые биты, перестроим деревья $A$ и $B$. Мы удалим вершины, которые запрашивают известные нам биты и подвесим детей в которых мы бы пошли по соответствующему вершине биту к родителю вершины. Либо $d_0$, либо $d_1$ уменьшится на $1$. После $2D^{dt}_1$ итераций у нас $d_0$ и $d_1$ должны стать равны 0, т.е. во всех оставшихся путях одни нолики. Но такой путь один, это просто лист. Он соответствует 0 или 1, выведем метку листа
%     \end{proof}
% \end{theorem}
% Здесь автоматически генерируется библиография. Первая команда задает стиль оформления библиографии, а вторая указывает на имя файла с расширением bib, в котором находится информация об источниках.
\bibliographystyle{plainurl}
\bibliography{bibl}




% % С этого момента глобальная нумерация идет буквами. Этот раздел я добавил лишь для демонстрации возможностей LaTeX, его можно и нужно удалить и он не нужен для курсового проекта непосредственно.
% \appendix

% Проведем небольшой обзор возможностей \LaTeX. Далее идет обзорный кусок, который надо будет вырезать. Он приведен лишь для демонстрации возможностей \LaTeX.

% \section{Нумеруемый заголовок}
% Текст раздела
% \subsection{Нумеруемый подзаголовок}
% Текст подраздела
% \subsubsection{Нумеруемый подподзаголовок}
% Текст подподраздела

% \section*{Не нумеруемый заголовок}
% Текст раздела
% \subsection*{Не нумеруемый подзаголовок}
% Текст подраздела
% \subsubsection*{Не нумеруемый подподзаголовок}
% Текст подподраздела


% \paragraph{Заголовок абзаца} Текст абзаца

% Формулы в тексте набирают так $x = e^{\pi i}\sqrt{\text{формула}}$. Выключенные не нумерованные формулы набираются либо так:
% \[
% x = e^{\pi i}\sqrt{\text{формула}}
% \]
% Либо так
% $$
% x = e^{\pi i}\sqrt{\text{формула}}
% $$
% Первый способ предпочтительнее при подаче статей в журналы AMS, потому рекомендую привыкать к нему.

% Выключенные нумерованные формулы:
% \begin{equation}\label{Equation1}
% % \label{имя-метки} эта команда ставит метку, на которую потом можно сослаться с помощью \ref{имя-метки}. Метки можно ставить на все объекты, у которых есть автоматические счетчики (номера разделов, подразделов, теорем, лемм, формул и т.д.
% x = e^{\pi i}\sqrt{\text{формула}}
% \end{equation}
% Или не нумерованная версия
% \begin{equation*}
% x = e^{\pi i}\sqrt{\text{формула}}
% \end{equation*}

% Уравнение~\ref{Equation1} радостно занумеровано.

% Лесенка для длинных формул
% \begin{multline}
% x = e^{\pi i}\sqrt{\text{очень очень очень длинная формула}}=\\
% \tr A - \sin(\text{еще одна очень очень длинная формула})=\\
% \cos z \Im \varphi(\text{и последняя длинная при длинная формула})
% \end{multline}

% Многострочная формула с центровкой
% \begin{gather}
% x = e^{\pi i}\sqrt{\text{очень очень очень длинная формула}}=\\
% \tr A - \sin(\text{еще одна очень очень длинная формула})=\\
% \cos z \Im \varphi(\text{и последняя длинная при длинная формула})
% \end{gather}

% Многострочная формула с ручным выравниванием. Выравнивание идет по знаку $\&$, который на печать не выводится.
% \begin{align}
% x = &e^{\pi i}\sqrt{\text{очень очень очень длинная формула}}=\\
% &\tr A - \sin(\text{еще одна очень очень длинная формула})=\\
% &\cos z \Im \varphi(\text{и последняя длинная при длинная формула})
% \end{align}

% \begin{theorem}
% Текст теоремы
% \end{theorem}
% \begin{proof}
% В специальном окружении оформляется доказательство.
% \end{proof}

% \begin{theorem}[Имя теоремы]
% Текст теоремы
% \end{theorem}
% \begin{proof}[Доказательство нашей теоремы]
% В специальном окружении оформляется доказательство.
% \end{proof}

% \begin{definition}
% Текст определения
% \end{definition}

% \begin{remark}
% Текст замечания
% \end{remark}

% \paragraph{Перечни:} Нумерованные
% \begin{enumerate}
% \item Первый
% \item Второй
% \begin{enumerate}
% \item Вложенный первый
% \item Вложенный второй
% \end{enumerate}
% \end{enumerate}

% Не нумерованные

% \begin{itemize}
% \item Первый
% \item Второй
% \begin{itemize}
% \item Вложенный первый
% \item Вложенный второй
% \end{itemize}
% \end{itemize}


% Здесь текст документа заканчивается
\end{document}
% Начиная с этого момента весь текст LaTeX игнорирует, можете вставлять любую абракадабру.
